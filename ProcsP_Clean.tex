CREATE PROCEDURE [dbo].[Agents_Insert]
(	
	@FirstName varchar(25),
	@LastName varchar(25),
	@Street varchar(25),
	@City varchar(25),
	@AgentState char(2),
	@Zip int,
	@BusPhone varchar(13),
	@CellPhone varchar(13),
	@Email varchar(75),
	@LicenseType1 varchar(25),
	@Type1Expiration varchar(25),
	@LicenseType2 varchar(25),
	@Type2Expiration varchar(25),
	@HireDate Date,
	@TerminationDate Date
) AS BEGIN
INSERT INTO [dbo].[Agents]
			([FirstName],[LastName],[Street],[City],[AgentState],[Zip],[BusPhone],[CellPhone],[Email],[LicenseType1],[Type1Expiration], [LicenseType2],[Type2Expiration],[HireDate],[TerminationDate])
VALUES
			(@Firstname, @Lastname, @Street, @City, @AgentState,@Zip,@BusPhone,@CellPhone,@Email,@LicenseType1,@Type1Expiration,@LicenseType2,@Type2Expiration,@HireDate,@TerminationDate)
			IF @@ROWCOUNT= 1
				RETURN 1
			ELSE
				RETURN 0
END
CREATE PROCEDURE [dbo].[INS]
(	
	@AgentFirstName varchar(25),
	@AgentLastName varchar(25),
	@AgentStreet varchar(25),
	@AgentCity varchar(25),
	@AgentState char(2),
	@AgentZip int,
	@AgentBusPhone varchar(13),
	@AgentCellPhone varchar(13),
	@AgentEmail varchar(75),
	@LicenseType1 varchar(25),
	@Type1Expiration varchar(25),
	@LicenseType2 varchar(25),
	@Type2Expiration varchar(25),
	@HireDate Date,
	@TerminationDate Date,
	--Contacts
	@Title varchar(25),
    @ContactFirstName varchar(35),
    @ContactLastName varchar(35),
    @BusinessName varchar(45),
    @NumEmployees int,
    @ContactType varchar(25),
    @ContactStreet varchar(35),
    @ContactCity varchar(25),
    @ContactState char(2),
    @ContactZip varchar(25),
    @ContactBusPhone varchar(13),
    @ContactHomePhone varchar(13),
    @ContactCellPhone varchar(13),
    @Fax varchar(25),
    @ContactEmail varchar(75),
    @Source varchar(45),
    @Website varchar(200),
    @Status varchar(25),
	--Communications
	@CommunicationType varchar(25),	
	@Outcome varchar(25),
	@CommunicationDate DATE,
	@CommunicationTime varchar(25),
	@FollowUp DATE,
	@AgentID int,
	@ContactID int
) AS BEGIN
INSERT INTO Agents([Agents].[FirstName],[Agents].[LastName],[Agents].[Street],[Agents].[City],[Agents].[AgentState],[Agents].[Zip],[Agents].[BusPhone],[Agents].[CellPhone],[Agents].[Email],[Agents].[LicenseType1],[Agents].[Type1Expiration],[Agents].[LicenseType2],[Agents].[Type2Expiration],[Agents].[HireDate],[Agents].[TerminationDate])			
	VALUES(		   @AgentFirstName,     @AgentLastName,     @AgentStreet,     @AgentCity,     @AgentState,          @AgentZip,     @AgentBusPhone,     @AgentCellPhone,     @AgentEmail,     @LicenseType1,          @Type1Expiration,          @LicenseType2,          @Type2Expiration,          @HireDate,          @TerminationDate);
INSERT INTO Contacts([Contacts].[Title],[Contacts].[FirstName],[Contacts].[LastName],[Contacts].[BusinessName],[Contacts].[NumEmployees],[Contacts].[ContactType],[Contacts].[Street],[Contacts].[City],[Contacts].[ContactState],[Contacts].[Zip],[Contacts].[BusPhone],[Contacts].[HomePhone],[Contacts].[CellPhone],[Contacts].[Fax],[Contacts].[Email],[Contacts].[Source],[Contacts].[Website],[Contacts].[Status])	
	VALUES(			 @Title,	        @ContactFirstName,	   @ContactLastName,     @BusinessName,	           @NumEmployees,            @ContactType,            @ContactStreet,     @ContactCity,     @ContactState,            @ContactZip ,    @ContactBusPhone,     @ContactHomePhone,     @ContactCellPhone,     @Fax,            @ContactEmail,     @Source,            @Website,            @Status);        
INSERT INTO	Communications([Communications].[CommunicationType],[Communications].[Outcome],[Communications].[CommunicationDate],[Communications].[CommunicationTime],[Communications].[FollowUp])	
	VALUES(				   @CommunicationType,                  @Outcome,                  @CommunicationDate,                  @CommunicationTime,                  @FollowUp);             			
			IF @@ROWCOUNT= 1
				RETURN 1
			ELSE
				RETURN 0
END
CREATE PROCEDURE [dbo].[INS_TEST](	
	@AgentFirstName varchar(25),
	@AgentLastName varchar(25),
	@AgentStreet varchar(25),
	@AgentCity varchar(25),
	@AgentState char(2),
	@AgentZip int,
	@AgentBusPhone varchar(13),
	@AgentCellPhone varchar(13),
	@AgentEmail varchar(75),
	@LicenseType1 varchar(25),
	@Type1Expiration varchar(25),
	@LicenseType2 varchar(25),
	@Type2Expiration varchar(25),
	@HireDate Date,
	@TerminationDate Date,
						--Contacts
	@Title varchar(25),
    @ContactFirstName varchar(35),
    @ContactLastName varchar(35),
    @BusinessName varchar(45),
    @NumEmployees int,
    @ContactType varchar(25),
    @ContactStreet varchar(35),
    @ContactCity varchar(25),
    @ContactState char(2),
    @ContactZip varchar(25),
    @ContactBusPhone varchar(13),
    @ContactHomePhone varchar(13),
    @ContactCellPhone varchar(13),
    @Fax varchar(25),
    @ContactEmail varchar(75),
    @Source varchar(45),
    @Website varchar(200),
    @Status varchar(25),
	@CommunicationID int,
					--Communications
	@CommunicationType varchar(25),	
	@Outcome varchar(25),
	@CommunicationDate DATE,
	@CommunicationTime varchar(25),
	@FollowUp DATE,
	@AgentID int,
	@ContactID int) AS BEGIN
DECLARE @Existing as int = 0
SELECT @Existing = Count(AgentID)
FROM Agents
WHERE AgentID = @AgentID
IF @Existing > 0 
BEGIN RAISERROR('ID Already Exists',1,1)
	RETURN 0 
END 
	INSERT INTO Agents([Agents].[AgentID],[Agents].[FirstName],[Agents].[LastName],[Agents].[Street],[Agents].[City],[Agents].[AgentState],[Agents].[Zip],[Agents].[BusPhone],[Agents].[CellPhone],[Agents].[Email],[Agents].[LicenseType1],[Agents].[Type1Expiration],[Agents].[LicenseType2],[Agents].[Type2Expiration],[Agents].[HireDate],[Agents].[TerminationDate])			
		VALUES(@AgentID,		   @AgentFirstName,     @AgentLastName,     @AgentStreet,     @AgentCity,     @AgentState,          @AgentZip,     @AgentBusPhone,     @AgentCellPhone,     @AgentEmail,     @LicenseType1,          @Type1Expiration,          @LicenseType2,          @Type2Expiration,          @HireDate,          @TerminationDate)
	IF @@ROWCOUNT= 1
				RETURN 1
			ELSE
				RETURN 0;
SET @Existing  = 0
SELECT @Existing = Count(ContactID)
FROM Contacts
WHERE ContactID = @ContactID
IF @Existing > 0 BEGIN
	RAISERROR('ID Already Exists',1,1)
	RETURN 0
END
INSERT INTO Contacts([Contacts].[ContactID],[Contacts].[Title],[Contacts].[FirstName],[Contacts].[LastName],[Contacts].[BusinessName],[Contacts].[NumEmployees],[Contacts].[ContactType],[Contacts].[Street],[Contacts].[City],[Contacts].[ContactState],[Contacts].[Zip],[Contacts].[BusPhone],[Contacts].[HomePhone],[Contacts].[CellPhone],[Contacts].[Fax],[Contacts].[Email],[Contacts].[Source],[Contacts].[Website],[Contacts].[Status])	
	VALUES(@ContactID,			 @Title,	        @ContactFirstName,	   @ContactLastName,     @BusinessName,	           @NumEmployees,            @ContactType,            @ContactStreet,     @ContactCity,     @ContactState,            @ContactZip ,    @ContactBusPhone,     @ContactHomePhone,     @ContactCellPhone,     @Fax,            @ContactEmail,     @Source,            @Website,            @Status)        
	IF @@ROWCOUNT= 1
				RETURN 1
			ELSE
				RETURN 0;
SET @Existing  = 0
SELECT @Existing = Count(CommunicationID)
FROM Communications
WHERE CommunicationID = @CommunicationID
IF @Existing > 0 
BEGIN
	RAISERROR('ID Already Exists',1,1)
	RETURN 0
END
INSERT INTO	Communications([Communications].[CommunicationID],[Communications].[CommunicationType],[Communications].[Outcome],[Communications].[CommunicationDate],[Communications].[CommunicationTime],[Communications].[FollowUp],[Communications].[AgentID],[Communications].[ContactID])	
	VALUES(@CommunicationID,				   @CommunicationType,                  @Outcome,                  @CommunicationDate,                  @CommunicationTime,                  @FollowUp,					 @AgentID,		            @ContactID)             			
			IF @@ROWCOUNT= 1
				RETURN 1
			ELSE
				RETURN 0;
END
CREATE PROCEDURE [dbo].[INS3]
(	
	@AgentFirstName varchar(25),
	@AgentLastName varchar(25),
	@AgentStreet varchar(25),
	@AgentCity varchar(25),
	@AgentState char(2),
	@AgentZip int,
	@AgentBusPhone varchar(13),
	@AgentCellPhone varchar(13),
	@AgentEmail varchar(75),
	@LicenseType1 varchar(25),
	@Type1Expiration varchar(25),
	@LicenseType2 varchar(25),
	@Type2Expiration varchar(25),
	@HireDate Date,
	@TerminationDate Date,
	--Contacts
	@Title varchar(25),
    @ContactFirstName varchar(35),
    @ContactLastName varchar(35),
    @BusinessName varchar(45),
    @NumEmployees int,
    @ContactType varchar(25),
    @ContactStreet varchar(35),
    @ContactCity varchar(25),
    @ContactState char(2),
    @ContactZip varchar(25),
    @ContactBusPhone varchar(13),
    @ContactHomePhone varchar(13),
    @ContactCellPhone varchar(13),
    @Fax varchar(25),
    @ContactEmail varchar(75),
    @Source varchar(45),
    @Website varchar(200),
    @Status varchar(25),
	--Communications
	@CommunicationType varchar(25),	
	@Outcome varchar(25),
	@CommunicationDate DATE,
	@CommunicationTime varchar(25),
	@FollowUp DATE,
	@AgentID int,
	@ContactID int
) AS BEGIN
INSERT INTO Agents([Agents].[FirstName],[Agents].[LastName],[Agents].[Street],[Agents].[City],[Agents].[AgentState],[Agents].[Zip],[Agents].[BusPhone],[Agents].[CellPhone],[Agents].[Email],[Agents].[LicenseType1],[Agents].[Type1Expiration],[Agents].[LicenseType2],[Agents].[Type2Expiration],[Agents].[HireDate],[Agents].[TerminationDate])			
	VALUES(		   @AgentFirstName,     @AgentLastName,     @AgentStreet,     @AgentCity,     @AgentState,          @AgentZip,     @AgentBusPhone,     @AgentCellPhone,     @AgentEmail,     @LicenseType1,          @Type1Expiration,          @LicenseType2,          @Type2Expiration,          @HireDate,          @TerminationDate);
	IF @@ROWCOUNT= 1
				RETURN 1
			ELSE
				RETURN 0
INSERT INTO Contacts([Contacts].[Title],[Contacts].[FirstName],[Contacts].[LastName],[Contacts].[BusinessName],[Contacts].[NumEmployees],[Contacts].[ContactType],[Contacts].[Street],[Contacts].[City],[Contacts].[ContactState],[Contacts].[Zip],[Contacts].[BusPhone],[Contacts].[HomePhone],[Contacts].[CellPhone],[Contacts].[Fax],[Contacts].[Email],[Contacts].[Source],[Contacts].[Website],[Contacts].[Status])	
	VALUES(			 @Title,	        @ContactFirstName,	   @ContactLastName,     @BusinessName,	           @NumEmployees,            @ContactType,            @ContactStreet,     @ContactCity,     @ContactState,            @ContactZip ,    @ContactBusPhone,     @ContactHomePhone,     @ContactCellPhone,     @Fax,            @ContactEmail,     @Source,            @Website,            @Status);        
	IF @@ROWCOUNT= 1
				RETURN 1
			ELSE
				RETURN 0
INSERT INTO	Communications([Communications].[CommunicationType],[Communications].[Outcome],[Communications].[CommunicationDate],[Communications].[CommunicationTime],[Communications].[FollowUp])	
	VALUES(				   @CommunicationType,                  @Outcome,                  @CommunicationDate,                  @CommunicationTime,                  @FollowUp);             			
			IF @@ROWCOUNT= 1
				RETURN 1
			ELSE
				RETURN 0
END
CREATE PROCEDURE [dbo].[INSERT Agents_Contacts_Communications]
(	
	@AgentFirstName varchar(25),
	@AgentLastName varchar(25),
	@AgentStreet varchar(25),
	@AgentCity varchar(25),
	@AgentState char(2),
	@AgentZip int,
	@AgentBusPhone varchar(13),
	@AgentCellPhone varchar(13),
	@AgentEmail varchar(75),
	@LicenseType1 varchar(25),
	@Type1Expiration varchar(25),
	@LicenseType2 varchar(25),
	@Type2Expiration varchar(25),
	@HireDate Date,
	@TerminationDate Date,
	--Contacts
	@Title varchar(25),
    @ContactFirstName varchar(35),
    @ContactLastName varchar(35),
    @BusinessName varchar(45),
    @NumEmployees int,
    @ContactType varchar(25),
    @ContactStreet varchar(35),
    @ContactCity varchar(25),
    @ContactState char(2),
    @ContactZip varchar(25),
    @ContactBusPhone varchar(13),
    @ContactHomePhone varchar(13),
    @ContactCellPhone varchar(13),
    @Fax varchar(25),
    @ContactEmail varchar(75),
    @Source varchar(45),
    @Website varchar(200),
    @Status varchar(25),
	--Communications
	@CommunicationType varchar(25),	
	@Outcome varchar(25),
	@CommunicationDate DATE,
	@CommunicationTime varchar(25),
	@FollowUp DATE,
	@AgentID int,
	@ContactID int
) AS BEGIN
INSERT INTO Agents([Agents].[FirstName],[Agents].[LastName],[Agents].[Street],[Agents].[City],[Agents].[AgentState],[Agents].[Zip],[Agents].[BusPhone],[Agents].[CellPhone],[Agents].[Email],[Agents].[LicenseType1],[Agents].[Type1Expiration],[Agents].[LicenseType2],[Agents].[Type2Expiration],[Agents].[HireDate],[Agents].[TerminationDate])			
	VALUES(		   @AgentFirstName,     @AgentLastName,     @AgentStreet,     @AgentCity,     @AgentState,          @AgentZip,     @AgentBusPhone,     @AgentCellPhone,     @AgentEmail,     @LicenseType1,          @Type1Expiration,          @LicenseType2,          @Type2Expiration,          @HireDate,          @TerminationDate);
INSERT INTO Contacts([Contacts].[Title],[Contacts].[FirstName],[Contacts].[LastName],[Contacts].[BusinessName],[Contacts].[NumEmployees],[Contacts].[ContactType],[Contacts].[Street],[Contacts].[City],[Contacts].[ContactState],[Contacts].[Zip],[Contacts].[BusPhone],[Contacts].[HomePhone],[Contacts].[CellPhone],[Contacts].[Fax],[Contacts].[Email],[Contacts].[Source],[Contacts].[Website],[Contacts].[Status])	
	VALUES(			 @Title,	        @ContactFirstName,	   @ContactLastName,     @BusinessName,	           @NumEmployees,            @ContactType,            @ContactStreet,     @ContactCity,     @ContactState,            @ContactZip ,    @ContactBusPhone,     @ContactHomePhone,     @ContactCellPhone,     @Fax,            @ContactEmail,     @Source,            @Website,            @Status);        
INSERT INTO	Communications([Communications].[CommunicationType],[Communications].[Outcome],[Communications].[CommunicationDate],[Communications].[CommunicationTime],[Communications].[FollowUp])	
	VALUES(				   @CommunicationType,                  @Outcome,                  @CommunicationDate,                  @CommunicationTime,                  @FollowUp);             			
			IF @@ROWCOUNT= 1
				RETURN 1
			ELSE
				RETURN 0
END
CREATE PROCEDURE [dbo].[INSERT_Carriers]
(	@Name varchar(25),	
	@Street varchar(25),
	@City varchar(25),
	@CarrierState char(2),
	@Zip int,
	@Contact varchar(13),
	@BusPhone varchar(13),
	@CellPhone varchar(13),
	@Email varchar(75)
) AS BEGIN
INSERT INTO [dbo].[Carriers]
			([Name],[Street],[City],[CarrierState],[Zip],[Contact],[BusPhone],[CellPhone],[Email])
VALUES
			(@Name, @Street, @City, @CarrierState, @Zip, @Contact, @BusPhone, @CellPhone, @Email)
			IF @@ROWCOUNT= 1
				RETURN 1
			ELSE
				RETURN 0
END
CREATE PROCEDURE [dbo].[insert_new_Agent]
(
		    @FirstName varchar(25)
           ,@LastName varchar(25)
           ,@Street varchar(25)
           ,@City varchar(25)
           ,@AgentState char(2)
           ,@Zip int
           ,@BusPhone varchar(13)
           ,@CellPhone varchar(13)
           ,@Email varchar(75)
           ,@LicenseType1 varchar(25)
           ,@Type1Expiration varchar(25)
           ,@LicenseType2 varchar(25)
           ,@Type2Expiration varchar(25)
           ,@HireDate date
           ,@TerminationDate date) AS BEGIN 
	INSERT INTO [dbo].[Agents]
			([FirstName],[LastName],[Street],[City],[AgentState],[Zip],[BusPhone],[CellPhone],[Email],[LicenseType1],[Type1Expiration], [LicenseType2],[Type2Expiration],[HireDate],[TerminationDate])
VALUES
			(@Firstname, @Lastname, @Street, @City, @AgentState,@Zip,@BusPhone,@CellPhone,@Email,@LicenseType1,@Type1Expiration,@LicenseType2,@Type2Expiration,@HireDate,@TerminationDate)
END
CREATE PROCEDURE [dbo].[insert_new_BusinessProspect]
(
@BusName varchar(45)
,@FollowUp varchar(645)
,@BusSize int
,@Contactid int
) AS BEGIN
INSERT INTO [dbo].[BusinessProspects]
           ([BusName],[FollowUp],[BusSize],[Contactid])
     VALUES
           (@BusName ,@FollowUp ,@BusSize ,@Contactid )
END
CREATE PROCEDURE [dbo].[insert_new_Carrier]
(
		    @CarrierName varchar(25)
           ,@Street varchar(25)
           ,@City varchar(25)
           ,@CarrierState char(2)
           ,@Zip int
           ,@Contact varchar(13)
           ,@BusPhone varchar(13)
           ,@CellPhone varchar(13)
           ,@Email varchar(75)
		   ) AS BEGIN 
	   INSERT INTO [dbo].[Carriers]
	(
		[CarrierName],[Street],[City],[CarrierState],[Zip],[Contact],[BusPhone],[CellPhone],[Email]
	)
     VALUES
	(
		@CarrierName ,@Street ,@City ,@CarrierState ,@Zip ,@Contact ,@BusPhone ,@CellPhone ,@Email 
	)
END
CREATE PROCEDURE [dbo].[insert_new_Communication]
		(
		    @CommunicationType varchar(25)
           ,@Outcome varchar(25)
           ,@CommunicationDate date
           ,@CommunicationTime varchar(25)
           ,@FollowUp date
           ,@AgentID int
           ,@ContactID int
		) AS BEGIN	
INSERT INTO [dbo].[Communications]
	(
		[CommunicationType],[Outcome],[CommunicationDate],[CommunicationTime],[FollowUp],[AgentID],[ContactID]
	)
     VALUES
	(
		@CommunicationType, @Outcome, @CommunicationDate, @CommunicationTime, @FollowUp, @AgentID, @ContactID 
	)
END
CREATE PROCEDURE [dbo].[insert_new_Contact]
			(
			@Title varchar(25)
           ,@FirstName varchar(35)
           ,@LastName varchar(35)
           ,@BusinessName varchar(45)
           ,@NumEmployees int
           ,@ContactType varchar(25)
           ,@Street varchar(35)
           ,@City varchar(25)
           ,@ContactState char(2)
           ,@Zip int
           ,@BusPhone varchar(13)
           ,@HomePhone varchar(13)
           ,@CellPhone varchar(13)
           ,@Fax varchar(25)
           ,@Email varchar(75)
           ,@Source varchar(45)
           ,@Website varchar(200)
           ,@Status varchar(25)
		   ) AS BEGIN 
	   INSERT INTO [dbo].[Contacts]
           (
[Title],[FirstName],[LastName],[BusinessName],[NumEmployees],[ContactType],[Street],[City],[ContactState],[Zip],[BusPhone],[HomePhone],[CellPhone],[Fax],[Email],[Source],[Website],[Status]
		   )
     VALUES
           (
@Title,@FirstName,@LastName,@BusinessName,@NumEmployees,@ContactType,@Street,@City,@ContactState,@Zip,@BusPhone,@HomePhone,@CellPhone,@Fax,@Email,@Source,@Website,@Status 
		   )
END
CREATE PROCEDURE [dbo].[insert_new_Policy]
		(
		   @PolicyNumber varchar(10)
           ,@TypeContact varchar(25)
           ,@MonthlyCommission decimal(8,2)
           ,@CarrierID int
           ,@AgentID int
		) AS BEGIN 
	INSERT INTO [dbo].[Policies]
           (
		   [PolicyNumber],[TypeContact],[MonthlyCommission],[CarrierID],[AgentID]
		   )
     VALUES
           (
			@PolicyNumber, @TypeContact, @MonthlyCommission, @CarrierID, @AgentID 
		   )
END
CREATE PROCEDURE [dbo].[myTest](@param1 as int OUTPUT) AS BEGIN
	SELECT @param1
	SET @param1 = 27
	RETURN -1
END
CREATE PROC [dbo].[procedureCursor](@Agents CURSOR VARYING OUTPUT) AS BEGIN
	SET @agents = CURSOR FOR
	SELECT Firstname
	FROM Agents
	ORDER BY Firstname
	OPEN @agents
END
CREATE proc [dbo].[securityTest]
AS BEGIN
	SELECT Firstname,Lastname
	FROM dbo.agents
END
CREATE proc [dbo].[seeAgents]
AS BEGIN
	SELECT CONCAT(Firstname, + ' ' + Lastname) AS 'Agent List'
	FROM dbo.agents
END
CREATE PROC [dbo].[SelectAgentWithMailAddress]
AS BEGIN
SELECT dbo.formatAddress(street,city,agentstate,zip) FROM [dbo].[agents]
END
CREATE FUNCTION [dbo].[formatAddress]
(
@Street as varchar(50), @city as varchar(50), @state as varchar(50), @zip as varchar(50)) RETURNS varchar(255) AS BEGIN
	IF (@Street IS NULL OR
	@city IS NULL OR
	@state IS NULL OR
	@zip IS NULL)
	RETURN 'Incomplete Address'
SET @state = 
CASE @state
	WHEN 'AL' THEN 'Alabama'
WHEN 'AK' THEN 'Alaska'	
WHEN 'AZ' THEN 'Arizona'	
WHEN 'AR' THEN 'Arkansas'	
WHEN 'CA' THEN 'California'	
WHEN 'CO' THEN 'Colorado'
WHEN 'CT' THEN 'Connecticut'
WHEN 'DE' THEN 'Delaware'
WHEN 'FL' THEN 'Florida'
WHEN 'GA' THEN 'Georgia'
WHEN 'HI' THEN 'Hawaii'
WHEN 'ID' THEN 'Idaho'
WHEN 'IL' THEN 'Illinois'
WHEN 'IN' THEN 'Indiana'
WHEN 'IA' THEN 'Iowa'
WHEN 'KS' THEN 'Kansas'
WHEN 'KY' THEN 'Kentucky'
WHEN 'LA' THEN 'Louisiana'
WHEN 'ME' THEN 'Maine'
WHEN 'MD' THEN 'Maryland'
WHEN 'MA' THEN 'Massachusetts'
WHEN 'MI' THEN 'Michigan'
WHEN 'MN' THEN 'Minnesota'
WHEN 'MS' THEN 'Mississippi'
WHEN 'MO' THEN 'Missouri'
WHEN 'MT' THEN 'Montana'
WHEN 'NE' THEN 'Nebraska'
WHEN 'NV' THEN 'Nevada'
WHEN 'NH' THEN 'New Hampshire'
WHEN 'NJ' THEN 'New Jersey'
WHEN 'NM' THEN 'New Mexico'
WHEN 'NY' THEN 'New York'
WHEN 'NC' THEN 'North Carolina'
WHEN 'ND' THEN 'North Dakota'
WHEN 'OH' THEN 'Ohio'
WHEN 'OK' THEN 'Oklahoma'
WHEN 'OR' THEN 'Oregon'
WHEN 'PA' THEN 'Pennsylvania'
WHEN 'RI' THEN 'Rhode Island'
WHEN 'SC' THEN 'South Carolina'
WHEN 'SD' THEN 'South Dakota'
WHEN 'TN' THEN 'Tennessee'
WHEN 'TX' THEN 'Texas'
WHEN 'UT' THEN 'Utah'
WHEN 'VT' THEN 'Vermont'
WHEN 'VA' THEN 'Virginia'
WHEN 'WA' THEN 'Washington'
WHEN 'WV' THEN 'West Virginia'
WHEN 'WI' THEN 'Wisconsin'
WHEN 'WY' THEN 'Wyoming'
END
RETURN @street + ' ' + @city +', ' + @state + ' ' + @zip
END
CREATE FUNCTION [dbo].[myTestFunction]() RETURNS int AS BEGIN
	return 'test'
end
	CREATE VIEW [dbo].[B_prospect_contactB] AS
	SELECT  BusinessProspects.BusProspectID, BusinessProspects.BusName, Contacts.BusinessName,Contacts.NumEmployees
	FROM BusinessProspects
	JOIN Contacts
	 ON BusinessProspects.Contactid = Contacts.Contactid
CREATE PROCEDURE [dbo].[Appointment_list] AS BEGIN
SELECT CONCAT(c.FirstName, c.LastName) as Contact, c.Homephone, c.cellphone, c.email,cm.communicationdate as'latest communication', cm.communicationTime as 'time of day', cm.followup as 'followup appointment'
  FROM  contacts as c
  JOIN communications as cm
		ON c.contactid = cm.contactid
WHERE cm.outcome IS NOT NULL 
ORDER BY (SELECT MIN(communications.communicationDate) FROM communications GROUP BY communications.followup) DESC
END
CREATE PROCEDURE [dbo].[Appointments]
@FirstName varchar(50),
@LastName varchar(50),
@BusinessName varchar(50),
@Outcome varchar(1000) AS BEGIN
SELECT [contacts].[FirstName], [contacts].[LastName], [contacts].[BusinessName],[Communications].[Outcome] FROM [Contacts]
JOIN [communications]
   ON [Contacts].[ContactID] = [Communications].[ContactID]
WHERE [Communications].[followup]  IS NOT NULL
END
CREATE PROCEDURE [dbo].[carriers_agents]
(
@lastname varchar (25)
) AS BEGIN
SELECT [agents].[agentid],[agents].[lastname] as 'Contact', [agents].[cellphone] as 'Cell', [carriers].[carrierName] as 'carrier', [agents].[email]
from [agents]
join [carriers]
	on [agents].[agentid] = [carriers].[carrierid]
	where [agents].[lastname] LIKE '	@LastName	'
END
CREATE PROCEDURE [dbo].[com_con] 
AS BEGIN
SELECT [Communications].[CommunicationDate],[Communications].[CommunicationTime],[communications].[outcome] 
FROM [Communications]
JOIN [Contacts]
	ON [communications].[contactid] = [contacts].[contactid]
WHERE [communications].[contactID] = [communications].[contactID]
END
CREATE PROC [dbo].[Contact_Full_Address]
AS BEGIN
SELECT [dbo].formatAddress(street,city,Contactstate,zip) FROM [dbo].[Contacts]
END
CREATE PROCEDURE [dbo].[ContactCarrierInfo]
(
@RepLastName varchar (25)
) AS BEGIN
SELECT [Carriers].[RepLastName] as 'Contact', [carriers].[carrierName] AS 'Provider', [Carriers].[BusPhone] AS 'Contact Phone Number', [Carriers].[Email]
FROM [Carriers]
	WHERE [Carriers].[RepLastName] = '@RepLastName'
END
CREATE PROCEDURE [dbo].[DaysBetween]
	(
		@CommunicationDate date,
		@CommunicationID date
	) AS BEGIN
--Count the days between two dates
SELECT datediff("d", date1, date2) AS days
FROM 
	(SELECT CommunicationDate AS date1
		FROM Communications
		WHERE CommunicationID = 147) t1,
	(SELECT CommunicationDate AS date2
		FROM Communications
		WHERE CommunicationID = 147) t2;
END
CREATE PROCEDURE [dbo].[insert_new_agent]			
(			
    @FirstName varchar(25)			
           ,@LastName varchar(25)			
           ,@Street varchar(25)			
           ,@City varchar(25)			
           ,@AgentState char(2)			
           ,@Zip int			
           ,@BusPhone varchar(13)			
           ,@CellPhone varchar(13)			
           ,@Email varchar(75)			
           ,@LicenseType1 varchar(25)			
           ,@Type1Expiration varchar(25)			
           ,@LicenseType2 varchar(25)			
           ,@Type2Expiration varchar(25)			
           ,@HireDate date			
           ,@TerminationDate date)			AS			BEGIN			
INSERT INTO [dbo].[Agents]			
			([FirstName],[LastName],[Street],[City],[AgentState],[Zip],[BusPhone],[CellPhone],[Email],[LicenseType1],[Type1Expiration], [LicenseType2],[Type2Expiration],[HireDate],[TerminationDate])
VALUES			
			(@Firstname, @Lastname, @Street, @City, @AgentState,@Zip,@BusPhone,@CellPhone,@Email,@LicenseType1,@Type1Expiration,@LicenseType2,@Type2Expiration,@HireDate,@TerminationDate)		
END
CREATE PROCEDURE [dbo].[insert_new_Carrier]			
(			
    @CarrierName varchar(25)			
           ,@Street varchar(25)			
           ,@City varchar(25)			
           ,@CarrierState char(2)			
           ,@Zip int			
           ,@Contact varchar(13)			
           ,@BusPhone varchar(13)			
           ,@CellPhone varchar(13)			
           ,@Email varchar(75)			
		   )		AS			BEGIN				
INSERT INTO [dbo].[Carriers]			
(			
[CarrierName],[Street],[City],[CarrierState],[Zip],[Contact],[BusPhone],[CellPhone],[Email]			
)			
     VALUES			
(			
@CarrierName ,@Street ,@City ,@CarrierState ,@Zip ,@Contact ,@BusPhone ,@CellPhone ,@Email 			
)			
END			
CREATE PROCEDURE [dbo].[insert_new_Communication]			
		(	
		    @CommunicationType varchar(25)	
           ,@Outcome varchar(25)			
           ,@CommunicationDate date			
           ,@CommunicationTime varchar(25)			
           ,@FollowUp date			
           ,@AgentID int			
           ,@ContactID int			
		)	AS			BEGIN				 	
	INSERT INTO [dbo].[Communications]			
(			
[CommunicationType],[Outcome],[CommunicationDate],[CommunicationTime],[FollowUp],[AgentID],[ContactID]			
)			
     VALUES			
(			
@CommunicationType, @Outcome, @CommunicationDate, @CommunicationTime, @FollowUp, @AgentID, @ContactID 			
)						
END			
CREATE PROCEDURE [dbo].[insert_new_Contact]			
			(
			@AgentID int
		   ,@Title varchar(25)
           ,@FirstName varchar(35)			
           ,@LastName varchar(35)			
           ,@BusinessName varchar(45)			
           ,@NumEmployees int			
           ,@ContactType varchar(25)			
           ,@Street varchar(35)			
           ,@City varchar(25)			
           ,@ContactState char(2)			
           ,@Zip int			
           ,@BusPhone varchar(13)			
           ,@HomePhone varchar(13)			
           ,@CellPhone varchar(13)			
           ,@Fax varchar(25)			
           ,@Email varchar(75)			
           ,@Source varchar(45)			
           ,@Website varchar(200)			
           ,@Status varchar(25)
		   ,@Notes varchar(1000)			
		   )	AS			BEGIN			
INSERT INTO [dbo].[Contacts]			
           (			
[AgentID],[Title],[FirstName],[LastName],[BusinessName],[NumEmployees],[ContactType],[Street],[City],[ContactState],[Zip],[BusPhone],[HomePhone],[CellPhone],[Fax],[Email],[Source],[Website],[Status],[Notes]			
		   )	
     VALUES			
           (			
@AgentID,@Title,@FirstName,@LastName,@BusinessName,@NumEmployees,@ContactType,@Street,@City,@ContactState,@Zip,@BusPhone,@HomePhone,@CellPhone,@Fax,@Email,@Source,@Website,@Status,@Notes 			
		   )		
END
CREATE PROCEDURE [dbo].[insert_new_Policy]
        (
		    @ContactID int
           ,@PolicyNumber varchar(10)
           ,@Carrier varchar(25)
           ,@Type varchar(25)
           ,@Commissions decimal(8,2)
           ,@CarrierID int
           ,@Subscribers int
		) AS BEGIN
	INSERT INTO [dbo].[Policies]
	(
	   [ContactID],[PolicyNumber],[Carrier],[Type],[Commissions],[CarrierID],[Subscribers]
	)
	 VALUES
		(
		    @ContactID
           ,@PolicyNumber 
           ,@Carrier 
           ,@Type 
           ,@Commissions 
           ,@CarrierID 
           ,@Subscribers
	    )
END
CREATE PROCEDURE [dbo].[NewComm]
			(
		   -----------params for Contacts
					@Notes varchar(1000)
					,@FirstName varchar(35)
					,@LastName varchar(35)
					,@BusPhone varchar(13)
					,@HomePhone varchar(13)
					,@CellPhone varchar(13)
					,@Street varchar(35)
					,@City varchar(25)
					,@ContactState char(2)
					,@Zip int
			) 
			AS
					INSERT INTO Contacts
			(
				[Contacts].[Notes]
				,[Contacts].[FirstName]
				,[Contacts].[LastName]
				,[Contacts].[BusPhone]
				,[Contacts].[HomePhone]
				,[Contacts].[CellPhone]
				,[Contacts].[Street]
				,[Contacts].[City]
				,[Contacts].[ContactState]
				,[Contacts].[Zip]
			)
	VALUES
			(	 
				@Notes 
				,@FirstName 
				,@LastName 
				,@BusPhone 
				,@HomePhone 
				,@CellPhone 
				,@Street
				,@City
				,@ContactState
				,@Zip
			)
CREATE PROCEDURE [dbo].[NewDeleteCommand]
(
	@Original_AgentID int,
	@IsNull_FirstName Int,
	@Original_FirstName varchar(25),
	@IsNull_LastName Int,
	@Original_LastName varchar(25),
	@IsNull_Street Int,
	@Original_Street varchar(25),
	@IsNull_City Int,
	@Original_City varchar(25),
	@IsNull_AgentState Int,
	@Original_AgentState char(2),
	@IsNull_Zip Int,
	@Original_Zip int,
	@IsNull_BusPhone Int,
	@Original_BusPhone varchar(13),
	@IsNull_CellPhone Int,
	@Original_CellPhone varchar(13),
	@IsNull_Email Int,
	@Original_Email varchar(75),
	@IsNull_LicenseType1 Int,
	@Original_LicenseType1 varchar(25),
	@IsNull_Type1Expiration Int,
	@Original_Type1Expiration varchar(25),
	@IsNull_LicenseType2 Int,
	@Original_LicenseType2 varchar(25),
	@IsNull_Type2Expiration Int,
	@Original_Type2Expiration varchar(25),
	@IsNull_HireDate Int,
	@Original_HireDate date,
	@IsNull_TerminationDate Int,
	@Original_TerminationDate date
) AS SET NOCOUNT OFF;
DELETE FROM [Agents] WHERE (([AgentID] = @Original_AgentID) AND ((@IsNull_FirstName = 1 AND [FirstName] IS NULL) OR ([FirstName] = @Original_FirstName)) AND ((@IsNull_LastName = 1 AND [LastName] IS NULL) OR ([LastName] = @Original_LastName)) AND ((@IsNull_Street = 1 AND [Street] IS NULL) OR ([Street] = @Original_Street)) AND ((@IsNull_City = 1 AND [City] IS NULL) OR ([City] = @Original_City)) AND ((@IsNull_AgentState = 1 AND [AgentState] IS NULL) OR ([AgentState] = @Original_AgentState)) AND ((@IsNull_Zip = 1 AND [Zip] IS NULL) OR ([Zip] = @Original_Zip)) AND ((@IsNull_BusPhone = 1 AND [BusPhone] IS NULL) OR ([BusPhone] = @Original_BusPhone)) AND ((@IsNull_CellPhone = 1 AND [CellPhone] IS NULL) OR ([CellPhone] = @Original_CellPhone)) AND ((@IsNull_Email = 1 AND [Email] IS NULL) OR ([Email] = @Original_Email)) AND ((@IsNull_LicenseType1 = 1 AND [LicenseType1] IS NULL) OR ([LicenseType1] = @Original_LicenseType1)) AND ((@IsNull_Type1Expiration = 1 AND [Type1Expiration] IS NULL) OR ([Type1Expiration] = @Original_Type1Expiration)) AND ((@IsNull_LicenseType2 = 1 AND [LicenseType2] IS NULL) OR ([LicenseType2] = @Original_LicenseType2)) AND ((@IsNull_Type2Expiration = 1 AND [Type2Expiration] IS NULL) OR ([Type2Expiration] = @Original_Type2Expiration)) AND ((@IsNull_HireDate = 1 AND [HireDate] IS NULL) OR ([HireDate] = @Original_HireDate)) AND ((@IsNull_TerminationDate = 1 AND [TerminationDate] IS NULL) OR ([TerminationDate] = @Original_TerminationDate)))
CREATE PROCEDURE [dbo].[NewUpdateCommand]
(
	@FirstName varchar(25),
	@LastName varchar(25),
	@Street varchar(25),
	@City varchar(25),
	@AgentState char(2),
	@Zip int,
	@BusPhone varchar(13),
	@CellPhone varchar(13),
	@Email varchar(75),
	@LicenseType1 varchar(25),
	@Type1Expiration varchar(25),
	@LicenseType2 varchar(25),
	@Type2Expiration varchar(25),
	@HireDate date,
	@TerminationDate date,
	@Original_AgentID int,
	@IsNull_FirstName Int,
	@Original_FirstName varchar(25),
	@IsNull_LastName Int,
	@Original_LastName varchar(25),
	@IsNull_Street Int,
	@Original_Street varchar(25),
	@IsNull_City Int,
	@Original_City varchar(25),
	@IsNull_AgentState Int,
	@Original_AgentState char(2),
	@IsNull_Zip Int,
	@Original_Zip int,
	@IsNull_BusPhone Int,
	@Original_BusPhone varchar(13),
	@IsNull_CellPhone Int,
	@Original_CellPhone varchar(13),
	@IsNull_Email Int,
	@Original_Email varchar(75),
	@IsNull_LicenseType1 Int,
	@Original_LicenseType1 varchar(25),
	@IsNull_Type1Expiration Int,
	@Original_Type1Expiration varchar(25),
	@IsNull_LicenseType2 Int,
	@Original_LicenseType2 varchar(25),
	@IsNull_Type2Expiration Int,
	@Original_Type2Expiration varchar(25),
	@IsNull_HireDate Int,
	@Original_HireDate date,
	@IsNull_TerminationDate Int,
	@Original_TerminationDate date,
	@AgentID int
) AS SET NOCOUNT OFF;
UPDATE [Agents] SET [FirstName] = @FirstName, [LastName] = @LastName, [Street] = @Street, [City] = @City, [AgentState] = @AgentState, [Zip] = @Zip, [BusPhone] = @BusPhone, [CellPhone] = @CellPhone, [Email] = @Email, [LicenseType1] = @LicenseType1, [Type1Expiration] = @Type1Expiration, [LicenseType2] = @LicenseType2, [Type2Expiration] = @Type2Expiration, [HireDate] = @HireDate, [TerminationDate] = @TerminationDate WHERE (([AgentID] = @Original_AgentID) AND ((@IsNull_FirstName = 1 AND [FirstName] IS NULL) OR ([FirstName] = @Original_FirstName)) AND ((@IsNull_LastName = 1 AND [LastName] IS NULL) OR ([LastName] = @Original_LastName)) AND ((@IsNull_Street = 1 AND [Street] IS NULL) OR ([Street] = @Original_Street)) AND ((@IsNull_City = 1 AND [City] IS NULL) OR ([City] = @Original_City)) AND ((@IsNull_AgentState = 1 AND [AgentState] IS NULL) OR ([AgentState] = @Original_AgentState)) AND ((@IsNull_Zip = 1 AND [Zip] IS NULL) OR ([Zip] = @Original_Zip)) AND ((@IsNull_BusPhone = 1 AND [BusPhone] IS NULL) OR ([BusPhone] = @Original_BusPhone)) AND ((@IsNull_CellPhone = 1 AND [CellPhone] IS NULL) OR ([CellPhone] = @Original_CellPhone)) AND ((@IsNull_Email = 1 AND [Email] IS NULL) OR ([Email] = @Original_Email)) AND ((@IsNull_LicenseType1 = 1 AND [LicenseType1] IS NULL) OR ([LicenseType1] = @Original_LicenseType1)) AND ((@IsNull_Type1Expiration = 1 AND [Type1Expiration] IS NULL) OR ([Type1Expiration] = @Original_Type1Expiration)) AND ((@IsNull_LicenseType2 = 1 AND [LicenseType2] IS NULL) OR ([LicenseType2] = @Original_LicenseType2)) AND ((@IsNull_Type2Expiration = 1 AND [Type2Expiration] IS NULL) OR ([Type2Expiration] = @Original_Type2Expiration)) AND ((@IsNull_HireDate = 1 AND [HireDate] IS NULL) OR ([HireDate] = @Original_HireDate)) AND ((@IsNull_TerminationDate = 1 AND [TerminationDate] IS NULL) OR ([TerminationDate] = @Original_TerminationDate)));
SELECT AgentID, FirstName, LastName, Street, City, AgentState, Zip, BusPhone, CellPhone, Email, LicenseType1, Type1Expiration, LicenseType2, Type2Expiration, HireDate, TerminationDate FROM Agents WHERE (AgentID = @AgentID)
CREATE PROCEDURE [dbo].[randomInteger] AS BEGIN
DECLARE @Random INT;
DECLARE @Upper INT;
DECLARE @Lower INT
SET @Lower = 1 
SET @Upper = 9 
SELECT @Random = ROUND(((@Upper - @Lower -1) * RAND() + @Lower), 0)
SELECT @Random
END
CREATE PROCEDURE [dbo].[SELECT_agent_IDENTITY]
AS
	SET NOCOUNT ON;
SELECT AgentID, FirstName, LastName, Street, City, AgentState, Zip, BusPhone, CellPhone, Email, LicenseType1, Type1Expiration, LicenseType2, Type2Expiration, HireDate, TerminationDate FROM Agents WHERE (AgentID = SCOPE_IDENTITY())
CREATE PROC [dbo].[SelectAgentWithMailAddress]
AS BEGIN
SELECT dbo.formatAddress(street,city,agentstate,zip) FROM [dbo].[agents]
END
CREATE FUNCTION [dbo].[formatAddress]
(@Street as varchar(50),
@city as varchar(50),
@state as varchar(50),
@zip as varchar(50)
)	RETURNS varchar(255) AS BEGIN
	IF (@Street IS NULL OR
	@city IS NULL OR
	@state IS NULL OR
	@zip IS NULL)
	RETURN 'Incomplete Address'
SET @state = 
CASE @state
	WHEN 'AL' THEN 'Alabama'
WHEN 'AK' THEN 'Alaska'	
WHEN 'AZ' THEN 'Arizona'	
WHEN 'AR' THEN 'Arkansas'	
WHEN 'CA' THEN 'California'	
WHEN 'CO' THEN 'Colorado'
WHEN 'CT' THEN 'Connecticut'
WHEN 'DE' THEN 'Delaware'
WHEN 'FL' THEN 'Florida'
WHEN 'GA' THEN 'Georgia'
WHEN 'HI' THEN 'Hawaii'
WHEN 'ID' THEN 'Idaho'
WHEN 'IL' THEN 'Illinois'
WHEN 'IN' THEN 'Indiana'
WHEN 'IA' THEN 'Iowa'
WHEN 'KS' THEN 'Kansas'
WHEN 'KY' THEN 'Kentucky'
WHEN 'LA' THEN 'Louisiana'
WHEN 'ME' THEN 'Maine'
WHEN 'MD' THEN 'Maryland'
WHEN 'MA' THEN 'Massachusetts'
WHEN 'MI' THEN 'Michigan'
WHEN 'MN' THEN 'Minnesota'
WHEN 'MS' THEN 'Mississippi'
WHEN 'MO' THEN 'Missouri'
WHEN 'MT' THEN 'Montana'
WHEN 'NE' THEN 'Nebraska'
WHEN 'NV' THEN 'Nevada'
WHEN 'NH' THEN 'New Hampshire'
WHEN 'NJ' THEN 'New Jersey'
WHEN 'NM' THEN 'New Mexico'
WHEN 'NY' THEN 'New York'
WHEN 'NC' THEN 'North Carolina'
WHEN 'ND' THEN 'North Dakota'
WHEN 'OH' THEN 'Ohio'
WHEN 'OK' THEN 'Oklahoma'
WHEN 'OR' THEN 'Oregon'
WHEN 'PA' THEN 'Pennsylvania'
WHEN 'RI' THEN 'Rhode Island'
WHEN 'SC' THEN 'South Carolina'
WHEN 'SD' THEN 'South Dakota'
WHEN 'TN' THEN 'Tennessee'
WHEN 'TX' THEN 'Texas'
WHEN 'UT' THEN 'Utah'
WHEN 'VT' THEN 'Vermont'
WHEN 'VA' THEN 'Virginia'
WHEN 'WA' THEN 'Washington'
WHEN 'WV' THEN 'West Virginia'
WHEN 'WI' THEN 'Wisconsin'
WHEN 'WY' THEN 'Wyoming'
END
RETURN @street + ' ' + @city +', ' + @state + ' ' + @zip
END
CREATE VIEW [dbo].[ALL_CONTACTS]
AS
SELECT * FROM CONTACTS
CREATE VIEW [dbo].[myView]
AS
SELECT Source
FROM CONTACTS
WHERE Source = 'Database'
CREATE VIEW [dbo].[Referrals]
AS
SELECT CONCAT(FirstName,LastName)ContactType,BusPhone,Source
FROM CONTACTS
WHERE Source = 'Referral'